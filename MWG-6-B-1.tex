%%%%%%%%%%%%%%%%%%%%%%%%%%%%%%%%%%%%%%%%%%%%%%%%%%%%%%%%%%%%%%%%%%%%%%
% How to use writeLaTeX:
%
% You edit the source code here on the left, and the preview on the
% right shows you the result within a few seconds.
%
% Bookmark this page and share the URL with your co-authors. They can
% edit at the same time!
%
% You can upload figures, bibliographies, custom classes and
% styles using the files menu.
%
% If you're new to LaTeX, the wikibook is a great place to start:
% http://en.wikibooks.org/wiki/LaTeX
%
%%%%%%%%%%%%%%%%%%%%%%%%%%%%%%%%%%%%%%%%%%%%%%%%%%%%%%%%%%%%%%%%%%%%%%
\documentclass{tufte-handout}

%\geometry{showframe}% for debugging purposes -- displays the margins

\usepackage{amsmath}
\usepackage{amssymb}

% Set up the images/graphics package
\usepackage{graphicx}
\setkeys{Gin}{width=\linewidth,totalheight=\textheight,keepaspectratio}
\graphicspath{{graphics/}}

\title{MWG 6.B.1}
\author{Bloody Micro! by Impatient Researcher}
%\date{}  % if the \date{} command is left out, the current date will be used

% The following package makes prettier tables.  We're all about the bling!
\usepackage{booktabs}

% The units package provides nice, non-stacked fractions and better spacing
% for units.
\usepackage{units}

% The fancyvrb package lets us customize the formatting of verbatim
% environments.  We use a slightly smaller font.
\usepackage{fancyvrb}
\fvset{fontsize=\normalsize}

% Small sections of multiple columns
\usepackage{multicol}

% Provides paragraphs of dummy text
\usepackage{lipsum}

% To highlight terms in equation
\usepackage{xcolor}

\newcommand{\highlight}[1]{%
  \colorbox{yellow!80}{$\displaystyle#1$}}

% These commands are used to pretty-print LaTeX commands
\newcommand{\doccmd}[1]{\texttt{\textbackslash#1}}% command name -- adds backslash automatically
\newcommand{\docopt}[1]{\ensuremath{\langle}\textrm{\textit{#1}}\ensuremath{\rangle}}% optional command argument
\newcommand{\docarg}[1]{\textrm{\textit{#1}}}% (required) command argument
\newenvironment{docspec}{\begin{quote}\noindent}{\end{quote}}% command specification environment
\newcommand{\docenv}[1]{\textsf{#1}}% environment name
\newcommand{\docpkg}[1]{\texttt{#1}}% package name
\newcommand{\doccls}[1]{\texttt{#1}}% document class name
\newcommand{\docclsopt}[1]{\texttt{#1}}% document class option name

\begin{document}

\maketitle% this prints the handout title, author, and date

%\printclassoptions

\section{Question}\label{sec:question}

Refer to the textbook for the actual wordings of the question. The main idea is to show that if Independent Axiom (IA) is satisfied, then:

$$L \succ L^{'} \iff \alpha L + (1 - \alpha) L^{''} \succ \alpha L^{'} + (1 - \alpha) L^{''}, \forall \alpha \in (0,1)$$

\noindent Where $L, L^{'}, L^{''} \in \mathcal{L}$.

\section{Answer}\label{sec:answer}

We will try to find a contradiction.

\begin{enumerate}

    \item Recall from the definition of IA:

    $$L \succcurlyeq L^{'} \iff \alpha L + (1 - \alpha) L^{''} \succcurlyeq \alpha L^{'} + (1 - \alpha) L^{''}, \forall \alpha \in (0,1)$$

    \item Suppose $L \succ L^{'}$. Notice that $L \succ L^{'} \implies L \succcurlyeq L^{'}$, IA applies! Thus

    $$\alpha L + (1 - \alpha) L^{''} \succcurlyeq \alpha L^{'} + (1 - \alpha) L^{''}, \forall \alpha \in (0,1)$$

    \item Then there are two distinct possibilities:
    \begin{enumerate}
      \item $\alpha L + (1 - \alpha) L^{''} \highlight{\sim} \alpha L^{'} + (1 - \alpha) L^{''}, \forall \alpha \in (0,1)$
      \item $\alpha L + (1 - \alpha) L^{''} \highlight{\succ} \alpha L^{'} + (1 - \alpha) L^{''}, \forall \alpha \in (0,1)$
    \end{enumerate}

    \item Suppose it is the indifference case, then it must be the case that\footnote{By the defintion of $\sim$, $$A \sim B \implies A \succcurlyeq B \text{ and } B \succcurlyeq A$$}:
    $$\alpha L^{'} + (1 - \alpha) L^{''} \succcurlyeq \alpha L^{} + (1 - \alpha) L^{''}, \forall \alpha \in (0,1)$$

    \item However, using IA again, this implies that $L^{'} \succcurlyeq L$. But this contradicts the claim that $L \succ L^{'}$. Thus it must be the case that $\alpha L + (1 - \alpha) L^{''} \highlight{\succ} \alpha L^{'} + (1 - \alpha) L^{''}, \forall \alpha \in (0,1)$. $\square$

\end{enumerate}

%\bibliography{sample-handout}
%\bibliographystyle{plainnat}

\end{document}
