%%%%%%%%%%%%%%%%%%%%%%%%%%%%%%%%%%%%%%%%%%%%%%%%%%%%%%%%%%%%%%%%%%%%%%
% How to use writeLaTeX: 
%
% You edit the source code here on the left, and the preview on the
% right shows you the result within a few seconds.
%
% Bookmark this page and share the URL with your co-authors. They can
% edit at the same time!
%
% You can upload figures, bibliographies, custom classes and
% styles using the files menu.
%
% If you're new to LaTeX, the wikibook is a great place to start:
% http://en.wikibooks.org/wiki/LaTeX
%
%%%%%%%%%%%%%%%%%%%%%%%%%%%%%%%%%%%%%%%%%%%%%%%%%%%%%%%%%%%%%%%%%%%%%%
\documentclass{tufte-handout}

%\geometry{showframe}% for debugging purposes -- displays the margins

\usepackage{amsmath}
\usepackage{amssymb}

% Set up the images/graphics package
\usepackage{graphicx}
\setkeys{Gin}{width=\linewidth,totalheight=\textheight,keepaspectratio}
\graphicspath{{graphics/}}

\title{Pure strategies in MNE must survive IEDS}
\author{Bloody Micro! by Impatient Researcher}
\date{20 December 2020}  % if the \date{} command is left out, the current date will be used

% The following package makes prettier tables.  We're all about the bling!
\usepackage{booktabs}

% The units package provides nice, non-stacked fractions and better spacing
% for units.
\usepackage{units}

% The fancyvrb package lets us customize the formatting of verbatim
% environments.  We use a slightly smaller font.
\usepackage{fancyvrb}
\fvset{fontsize=\normalsize}

% Small sections of multiple columns
\usepackage{multicol}

% Provides paragraphs of dummy text
\usepackage{lipsum}

% These commands are used to pretty-print LaTeX commands
\newcommand{\doccmd}[1]{\texttt{\textbackslash#1}}% command name -- adds backslash automatically
\newcommand{\docopt}[1]{\ensuremath{\langle}\textrm{\textit{#1}}\ensuremath{\rangle}}% optional command argument
\newcommand{\docarg}[1]{\textrm{\textit{#1}}}% (required) command argument
\newenvironment{docspec}{\begin{quote}\noindent}{\end{quote}}% command specification environment
\newcommand{\docenv}[1]{\textsf{#1}}% environment name
\newcommand{\docpkg}[1]{\texttt{#1}}% package name
\newcommand{\doccls}[1]{\texttt{#1}}% document class name
\newcommand{\docclsopt}[1]{\texttt{#1}}% document class option name

\begin{document}

\maketitle% this prints the handout title, author, and date

%\printclassoptions

\section{Question}\label{sec:question}

Formally show that in a Mixed Strategy Nash Equilibrium (MNE), pure strategies that would be played with a positive probability must survive Iterated Elimination of Dominated Strategies (IEDS).

\section{Answer}\label{sec:answer}

Suppose not\footnote{That is we have a MNE on hand, and there exists a pure strategy that gets played with positive probability but somehow it gets eliminated in the process.}. Consider a 2-person case. 

\begin{enumerate}
    
    \item Denote the Mixed Strategy Nash Equilibrium by $p^* = (p^*_1, p^*_2)$.
    
    \item There are $K$ strategies one can take, denote the strategy sets by $S_i = (s_{i1}, \dots, s_{iK}), \forall i \in \{1,2\}$
    
    \item WLOG, assume $s_{11}$\footnote{This reads the first strategy of the first player} be the first pure strategy which is:
        \begin{enumerate}
            \item being played with positive probability in a MNE; and
            \item get eliminated!
        \end{enumerate}
        This also implies that there must exist another strategy, $s_{1j}$ that:
        \begin{enumerate}
            \item has not yet been eliminated; and
            \item $u_1(s_{11}, s_{2k}) < u_1(s_{1j}, s_{2k}), \forall s_{2k} \in S_2$\footnote{'Cause that's how you get to eliminate $s_{11}$ in the first place!}
        \end{enumerate}
   
    \item But if $s_{11}$ is worse than some $s_{1j}$ regardless of what strategy player 2 plays, then $s_{11}$ is still worse than $s_{1j}$ when player 2 is \textit{randomising} over his/her strategy set $S_2$\footnote{Because by \textit{randomising}, we mean that player 2 is playing a convex combination of strategies $S_2 = (s_{21}, \dots, s_{2K})$}: $$u_1(s_{11}, p^*_2) < u_1(s_{1j}, p^*_2)$$
   
    \item Then we can generate a new mixed strategy for player 1 as follows:
    \begin{equation*}
        p^{**}_1 =
        \begin{cases}
            p^{**}_{11} = 0 & \text{Cut the prob. of playing } s_{11}   \text{ to } 0\\
            p^{**}_{1j} = p^{*}_{11} + p^{*}_{1j} & \text{Play the better } s_{1j} \text{ more often!}\\
            p^{**}_{1k} = p^{*}_{1k} & \forall k \neq {1, j} 
        \end{cases}       
    \end{equation*}
    
    \item By construction, $$u_1(p^{**}_{1}, p^{*}_{2}) > u_1(p^{*}_{1}, p^{*}_{2})$$ 
    But clearly, it contradicts the claim that $p^* = (p^*_1, p^*_2)$ is a MNE! $\square$
\end{enumerate}

%\bibliography{sample-handout}
%\bibliographystyle{plainnat}

\end{document}