%%%%%%%%%%%%%%%%%%%%%%%%%%%%%%%%%%%%%%%%%%%%%%%%%%%%%%%%%%%%%%%%%%%%%%
% How to use writeLaTeX: 
%
% You edit the source code here on the left, and the preview on the
% right shows you the result within a few seconds.
%
% Bookmark this page and share the URL with your co-authors. They can
% edit at the same time!
%
% You can upload figures, bibliographies, custom classes and
% styles using the files menu.
%
% If you're new to LaTeX, the wikibook is a great place to start:
% http://en.wikibooks.org/wiki/LaTeX
%
%%%%%%%%%%%%%%%%%%%%%%%%%%%%%%%%%%%%%%%%%%%%%%%%%%%%%%%%%%%%%%%%%%%%%%
\documentclass{tufte-handout}

%\geometry{showframe}% for debugging purposes -- displays the margins

\usepackage{amsmath}
\usepackage{amssymb}

% Set up the images/graphics package
\usepackage{graphicx}
\setkeys{Gin}{width=\linewidth,totalheight=\textheight,keepaspectratio}
\graphicspath{{graphics/}}

\title{Expected payment in standard auctions}
\author{Bloody Micro! by Impatient Researcher}
\date{23 December 2020}  % if the \date{} command is left out, the current date will be used

% The following package makes prettier tables.  We're all about the bling!
\usepackage{booktabs}

% The units package provides nice, non-stacked fractions and better spacing
% for units.
\usepackage{units}

% The fancyvrb package lets us customize the formatting of verbatim
% environments.  We use a slightly smaller font.
\usepackage{fancyvrb}
\fvset{fontsize=\normalsize}

% Small sections of multiple columns
\usepackage{multicol}

% Provides paragraphs of dummy text
\usepackage{lipsum}

% These commands are used to pretty-print LaTeX commands
\newcommand{\doccmd}[1]{\texttt{\textbackslash#1}}% command name -- adds backslash automatically
\newcommand{\docopt}[1]{\ensuremath{\langle}\textrm{\textit{#1}}\ensuremath{\rangle}}% optional command argument
\newcommand{\docarg}[1]{\textrm{\textit{#1}}}% (required) command argument
\newenvironment{docspec}{\begin{quote}\noindent}{\end{quote}}% command specification environment
\newcommand{\docenv}[1]{\textsf{#1}}% environment name
\newcommand{\docpkg}[1]{\texttt{#1}}% package name
\newcommand{\doccls}[1]{\texttt{#1}}% document class name
\newcommand{\docclsopt}[1]{\texttt{#1}}% document class option name

\begin{document}

\maketitle% this prints the handout title, author, and date

%\printclassoptions

\section{Question}\label{sec:question}

Derive the expected payment in the following two standard auctions\footnote{Standard auctions are simply auctions where the highest bidder gets the item.} under the assumption of Independent Private Valuation (IPV):
\begin{enumerate}
    \item First price auction
    \item Second price auction
\end{enumerate}

\section{Answer}\label{sec:answer}

Expected payment is the product of two things:
\begin{enumerate}
    \item How much I have to pay in the event I win
    \item The probability I win
\end{enumerate}

The details of the auction design is secondary.

\subsection{First price auction}

\begin{enumerate}
    
    \item Derive the optimal bidding strategy yourself. I will directly state the answer here\footnote{$Y_1$ is the second highest order statistic which is essentially the second highest valuation while acknowledging that it is a random variable. $r$ is the reserve price, if any.}: $$\beta^I(\theta) = E\left[ \max \left(Y_1, r \right) | Y_1 \leq \theta \right]$$

    \item The probability of winning is simply: $[F(\theta)]^{(N-1)}$ where $F$ is the CDF of everyone's valuation. To ease notation, the literature often denote $G(.) \equiv [F(.)]^{(N-1)}$.
    
    \item So we have $M^I(\theta) = G(\theta) \times E\left[ \max \left(Y_1, r \right) | Y_1 \leq \theta \right]$
    
    \item Notice $E\left[ \max \left(Y_1, r \right) | Y_1 \leq \theta \right] = \frac{1}{G(\theta)}\int_0^\theta max(y, r) g(y) dy$, which can be simplified much further when breaking the limits of integration into two intervals and evaluate $\max(.)$ in each of them\footnote{$\frac{1}{G(\theta)} \left[ \int_0^r max(y, r) g(y) dy + \int_r^\theta max(y, r) g(y) dy \right]$}. Ultimately, you will get:
    
    $$E\left[ \max \left(Y_1, r \right) | Y_1 \leq \theta \right] = \frac{1}{G(\theta)}\left[ rG(r) + \int_r^\theta y g(y) dy \right]$$
    
    \item The expected payment is therefore\footnote{The second equality comes from Integration By Parts.}:
    $$M^I(\theta) = G(\theta) \times \frac{1}{G(\theta)}\left[ rG(r) + \int_r^\theta y g(y) dy \right] = rG(r) + \int_r^\theta y g(y) dy$$
    
\end{enumerate}
\newpage


\subsection{Second price auction}
\begin{enumerate}
    \item Argue that bidding one's true valuation is a weakly dominant strategy.
    \item Thus in a symmetric equilibrium, everyone's truth telling and your winning probability is unsurprisingly $G(\theta)$.
    \item And in the event you win, you pay the second highest bid which is the second highest valuation\footnote{Remember when the individual submits a bid, he/she only knows his/her valuation and not others. Thus he/she has to guess what the second highest valuation might be based on the only piece of information he/she has, which is the CDF.} in the equilibrium: $$E\left[ \max \left(Y_1, r \right) | Y_1 \leq \theta \right]$$
    \item Therefore we have again:
    $$M^{II}(\theta) = \left[ rG(r) + \int_r^\theta y g(y) dy \right]$$
\end{enumerate}

\subsection{Remark on the second price auction}

In the second price setting, there is an alternative interpretation:

\begin{enumerate}

    \item When the $(n-1)$ highest order statistic is \textbf{smaller} than $r$, then the "second highest" price in the model is effectively the reserve price (Assuming of course $\theta$ is as high as $r$), and that's the price you have to pay when win.
    
    \item But if the $(n-1)$ highest order statistic is \textbf{higher} than $r$, then it's just business as usual, i.e. $(n-1)$ highest order statistic is the effective second highest price.

\end{enumerate}

%\bibliography{sample-handout}
%\bibliographystyle{plainnat}

\end{document}