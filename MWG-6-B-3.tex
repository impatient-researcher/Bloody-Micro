%%%%%%%%%%%%%%%%%%%%%%%%%%%%%%%%%%%%%%%%%%%%%%%%%%%%%%%%%%%%%%%%%%%%%%
% How to use writeLaTeX:
%
% You edit the source code here on the left, and the preview on the
% right shows you the result within a few seconds.
%
% Bookmark this page and share the URL with your co-authors. They can
% edit at the same time!
%
% You can upload figures, bibliographies, custom classes and
% styles using the files menu.
%
% If you're new to LaTeX, the wikibook is a great place to start:
% http://en.wikibooks.org/wiki/LaTeX
%
%%%%%%%%%%%%%%%%%%%%%%%%%%%%%%%%%%%%%%%%%%%%%%%%%%%%%%%%%%%%%%%%%%%%%%
\documentclass{tufte-handout}

%\geometry{showframe}% for debugging purposes -- displays the margins

\usepackage{amsmath}
\usepackage{amssymb}

% Set up the images/graphics package
\usepackage{graphicx}
\setkeys{Gin}{width=\linewidth,totalheight=\textheight,keepaspectratio}
\graphicspath{{graphics/}}

\title{MWG 6.B.3}
\author{Bloody Micro! by Impatient Researcher}
%\date{}  % if the \date{} command is left out, the current date will be used

% The following package makes prettier tables.  We're all about the bling!
\usepackage{booktabs}

% The units package provides nice, non-stacked fractions and better spacing
% for units.
\usepackage{units}

% The fancyvrb package lets us customize the formatting of verbatim
% environments.  We use a slightly smaller font.
\usepackage{fancyvrb}
\fvset{fontsize=\normalsize}

% Small sections of multiple columns
\usepackage{multicol}

% Provides paragraphs of dummy text
\usepackage{lipsum}

% To highlight terms in equation
\usepackage{xcolor}

\newcommand{\highlight}[1]{%
  \colorbox{yellow!80}{$\displaystyle#1$}}

% These commands are used to pretty-print LaTeX commands
\newcommand{\doccmd}[1]{\texttt{\textbackslash#1}}% command name -- adds backslash automatically
\newcommand{\docopt}[1]{\ensuremath{\langle}\textrm{\textit{#1}}\ensuremath{\rangle}}% optional command argument
\newcommand{\docarg}[1]{\textrm{\textit{#1}}}% (required) command argument
\newenvironment{docspec}{\begin{quote}\noindent}{\end{quote}}% command specification environment
\newcommand{\docenv}[1]{\textsf{#1}}% environment name
\newcommand{\docpkg}[1]{\texttt{#1}}% package name
\newcommand{\doccls}[1]{\texttt{#1}}% document class name
\newcommand{\docclsopt}[1]{\texttt{#1}}% document class option name

\begin{document}

\maketitle% this prints the handout title, author, and date

%\printclassoptions

\section{Question}\label{sec:question}

Refer to the textbook for the actual wordings of the question. The main idea is to show that if:

\begin{enumerate}
  \item the set of outcome is finite
  \item the individual's preference over the set of outcomes is \textit{rational}\footnote{That is preference is complete and transitive}
  \item the independence axiom is satisfied
\end{enumerate}

\noindent Then you can always find the best and the worst lottery.

\section{Answer}\label{sec:answer}

To prove this, we construct the best and the worst lottery directly.

\begin{enumerate}

    \item Denote the set of outcomes by $C$ and the rational relation by $\succcurlyeq$.

    \item As the set $C$ is finite, $\succcurlyeq$ is complete and transitive, there must exist at least one outome that is deemed to the best and and at least one deemed to be the worst.\footnote{$C$ being finite means there are just that many pairwise comparisons to be done, and rational preference implies that the individual can compare any pair of outcomes (completeness) and form a uncontradictory ordering of outcomes (transitivity).}

    \item WLOG, let the number of outcomes be $N$ such that:
    $$c_1 \succ c_2 \succ c_3 \succ \dots \succ c_N$$

    \item Let $L$ be a generic lottery which takes the form:
    $$L = \left( p_1 \circ c_1, p_2 \circ c_2, p_3 \circ c_3, \dots, p_N \circ c_N \right)$$

    \item We can make the lottery $L$ more attractive\footnote{This is because except $c_2$ all other items are the same. And by IA, $L' \succ L \iff c_1 \succ c_2$} by replacing $c_2$ by $c_1$:
    $$L' = \left( p_1 \circ c_1, p_2 \circ \highlight{c_1}, p_3 \circ c_3, \dots, p_N \circ c_N \right)$$

    \item Following this logic, we can create an even better lottery by successively replacing $c_3, \dots, c_N$ by $c_1$. Clearly, there's no room for improvement when we have replaced all outcomes in $L$ with $c_1$ that is:
    $$\overline{L} = \left( p_1 \circ c_1, p_2 \circ \highlight{c_1}, p_3 \circ \highlight{c_1}, \dots, p_N \circ \highlight{c_1} \right)$$

    Implying that $\overline{L}$ constructed this way is indeed the best lottery.

    \item Similarly one can construct the worst lottery $\underbar{L}$ by replacing all outcomes by $c_N$. $\square$

\end{enumerate}

%\bibliography{sample-handout}
%\bibliographystyle{plainnat}

\end{document}
