%%%%%%%%%%%%%%%%%%%%%%%%%%%%%%%%%%%%%%%%%%%%%%%%%%%%%%%%%%%%%%%%%%%%%%
% How to use writeLaTeX:
%
% You edit the source code here on the left, and the preview on the
% right shows you the result within a few seconds.
%
% Bookmark this page and share the URL with your co-authors. They can
% edit at the same time!
%
% You can upload figures, bibliographies, custom classes and
% styles using the files menu.
%
% If you're new to LaTeX, the wikibook is a great place to start:
% http://en.wikibooks.org/wiki/LaTeX
%
%%%%%%%%%%%%%%%%%%%%%%%%%%%%%%%%%%%%%%%%%%%%%%%%%%%%%%%%%%%%%%%%%%%%%%
\documentclass{tufte-handout}

%\geometry{showframe}% for debugging purposes -- displays the margins

\usepackage{amsmath}
\usepackage{amssymb}

% Set up the images/graphics package
\usepackage{graphicx}
\setkeys{Gin}{width=\linewidth,totalheight=\textheight,keepaspectratio}
\graphicspath{{graphics/}}

\title{Proposer optimal DA matching}
\author{Bloody Micro! by Impatient Researcher}
%\date{}  % if the \date{} command is left out, the current date will be used

% The following package makes prettier tables.  We're all about the bling!
\usepackage{booktabs}

% The units package provides nice, non-stacked fractions and better spacing
% for units.
\usepackage{units}

% The fancyvrb package lets us customize the formatting of verbatim
% environments.  We use a slightly smaller font.
\usepackage{fancyvrb}
\fvset{fontsize=\normalsize}

% Small sections of multiple columns
\usepackage{multicol}

% Provides paragraphs of dummy text
\usepackage{lipsum}

% To highlight terms in equation
\usepackage{xcolor}

\newcommand{\highlight}[1]{%
  \colorbox{yellow!80}{$\displaystyle#1$}}

% To use fancy curly alphabets in math equations
\usepackage{mathrsfs}

% These commands are used to pretty-print LaTeX commands
\newcommand{\doccmd}[1]{\texttt{\textbackslash#1}}% command name -- adds backslash automatically
\newcommand{\docopt}[1]{\ensuremath{\langle}\textrm{\textit{#1}}\ensuremath{\rangle}}% optional command argument
\newcommand{\docarg}[1]{\textrm{\textit{#1}}}% (required) command argument
\newenvironment{docspec}{\begin{quote}\noindent}{\end{quote}}% command specification environment
\newcommand{\docenv}[1]{\textsf{#1}}% environment name
\newcommand{\docpkg}[1]{\texttt{#1}}% package name
\newcommand{\doccls}[1]{\texttt{#1}}% document class name
\newcommand{\docclsopt}[1]{\texttt{#1}}% document class option name

\begin{document}

\maketitle% this prints the handout title, author, and date

%\printclassoptions

\section{Question}\label{sec:question}

Consider a two-sided matching problem. To make it concrete, let's consider a marriage problem where there is a group of male proposing to a group of female.

\bigskip

\noindent Show that:
\begin{enumerate}
    \item Male-proposing DA results in a male-optimal outcome
    \item Male-proposing DA results in a female-pessimal outcome
\end{enumerate}

\section{Answer}\label{sec:answer}

\subsection{Part 1}\label{sec:answer-part-one}

Our strategy is this - we first define what an optimality means for the male group and then we prove this by contradiction.

\bigskip

\noindent If a (stable) matching is male-optimal, then by definition there \textbf{cannot} exist another (stable) matching where a man is matched to another woman for whom he strictly prefers. This is a bit mouthful so let's summarise this criterion of optimality into: "no man is rejected by an achievable woman".\footnote{Succinctly: $w$ is achievable for some $m$ if there $\exists$ a stable $\mu$ such that $\mu(m) = w$}

\bigskip 

\noindent This notion of optimality is important in terms of how we structure the contradiction statement.

\begin{enumerate}

    \item Suppose not - the matching is not male-optimal, then by definition, there must exist at least one instance where a man is rejected by an achievable woman. 
    
    \item Consider the first man being rejected. Denote this man as $m$ and the woman rejecting him as $w$.\footnote{Our definition of achievable, again, is that there exists a stable matching $\mu'$ where $\mu'(m) = w$}
    
    \item At this very step, $w$ rejects $m$, she must be keeping some other man, call him $m'$.\footnote{Note this man may not be the ultimate match. And also as $\mu'(m) = w$ by supposition, $\mu'(m') \neq w$ as we cannot have $w$ matched to both $m$ and $m'$ simultaneously. Convention has it that we have $\mu'(m') = w'$}
    
    \item But supposition, this is the first step of the DA where a man is rejected by an achievable woman, then it must be the case that $m'$ has not been rejected by any woman. Also, $m'$ must first propose to the one he prefers the most\footnote{This is how male-proposing DA works.} - this implies that $w \succ_{m'} \mu'(m')$.
    
    \item Then by the fact that $w$ rejects $m$ for $m'$, it must be the case that $m' \succ_{w} \mu'(w)$. Thus $m'$ and $w$ form a blocking pair, implying that $\mu'(.)$ is not stable. $\square$ 
\end{enumerate}

\subsection{Part 2}\label{sec:answer-part-two}

\begin{enumerate}

    \item Suppose not - the matching is not female-pessimal, then there must exist a worse stable matching outcome, $\mu'(.)$ , for a at least one female.
    
    \item Pick one unfortunate woman who by supposition can have an even worse match. Denote her as $w$, then we have $\mu'(w) = m'$ and $m \succ_{w} m'$.
    
    \item We have shown previously that the male-proposing DA outcome is male-optimal, then we know for sure $w \succ_{m} w'$
    
    \item Again under $\mu'(.)$, $m$ and $w$ can form a blocking pair, contradicting the claim that $\mu'(.)$ is stable. $\square$
\end{enumerate}

%\bibliography{sample-handout}
%\bibliographystyle{plainnat}

\end{document}
